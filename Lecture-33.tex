\documentclass[10pt]{article}
\usepackage{NotesTeX} %/Path/to/package should be replaced with package location
\usepackage{lipsum}
\usepackage{tensor}
\usepackage{amsmath,amsthm,amssymb}
\usepackage{hyperref}
\usepackage{physics}
\input{undertilde}


\newcommand{\bs}{\textbackslash}


\title{{\Huge General Relativity}\\{\Large{Class  33 - APRIL BLANK, 2020}}} %replace with class number
\author{Sarah Racz}

\emailAdd{racz.sarah@utexas.edu} %replace with your email
\begin{document}
    \maketitle
    \flushbottom
    \newpage
    \pagestyle{fancynotes}
    %\part{HELLO \LaTeX\,}
	%Use the uncompiled version of this document in itself as a \LaTeX\, style guide for the class you'll be responsible for.
	
	\textbf{DRAFT 1 }
	
	\section{Null Hypersurfaces}
	In the previous lecture we discussed null hypersurfaces. Recall that for a tangent function to some hyper surface, for example $\xi^\mu = h(x^\alpha) g^{\mu \nu} \partial_nu f$, where $f$ is a function set to a constant $f(x^\alpha) = f_*$. The choice of the function $f_*$ selects which hypersurface $\Sigma$ we are on. $\Sigma$ is then a null hypersurface if $xi^\mu$  the tangent obeys  $\xi^\mu \xi_\mu = 0$. As discussed previously these tangents are always tangent to geodesics and the geodesics are called the geodesic generators of $\Sigma$. From now on we will use the convention where we choose $h(\alpha)$ such that $\xi^\mu$ is affinely parametrized. 
	
            \begin{figure}[h]
            \begin{center}            
            
            \tikzset{every picture/.style={line width=0.75pt}} %set default line width to 0.75pt        
            
            \begin{tikzpicture}[x=0.75pt,y=0.75pt,yscale=-1,xscale=1]
            %uncomment if require: \path (0,300); %set diagram left start at 0, and has height of 300
            
            %Shape: Parallelogram [id:dp4342085762949739] 
            \draw   (143.57,99.99) -- (248.76,100.01) -- (199.69,213.02) -- (94.5,213) -- cycle ;
            %Straight Lines [id:da2907024348368141] 
            \draw    (119.5,201) -- (164.59,112.78) ;
            \draw [shift={(165.5,111)}, rotate = 477.07] [color={rgb, 255:red, 0; green, 0; blue, 0 }  ][line width=0.75]    (10.93,-3.29) .. controls (6.95,-1.4) and (3.31,-0.3) .. (0,0) .. controls (3.31,0.3) and (6.95,1.4) .. (10.93,3.29)   ;
            %Straight Lines [id:da5214645927404655] 
            \draw    (145.5,201) -- (191.07,114.77) ;
            \draw [shift={(192,113)}, rotate = 477.85] [color={rgb, 255:red, 0; green, 0; blue, 0 }  ][line width=0.75]    (10.93,-3.29) .. controls (6.95,-1.4) and (3.31,-0.3) .. (0,0) .. controls (3.31,0.3) and (6.95,1.4) .. (10.93,3.29)   ;
            %Straight Lines [id:da2829743866608323] 
            \draw    (180.5,200) -- (221.6,118.78) ;
            \draw [shift={(222.5,117)}, rotate = 476.84] [color={rgb, 255:red, 0; green, 0; blue, 0 }  ][line width=0.75]    (10.93,-3.29) .. controls (6.95,-1.4) and (3.31,-0.3) .. (0,0) .. controls (3.31,0.3) and (6.95,1.4) .. (10.93,3.29)   ;
            %Curve Lines [id:da33663440486775986] 
            \draw    (197,123) .. controls (217.5,85) and (253.5,86) .. (277.5,92) ;
            
            % Text Node
            \draw (281,87.4) node [anchor=north west][inner sep=0.75pt]    {$x^{\mu }( \lambda ) \ $};
            % Text Node
            \draw (260,132.4) node [anchor=north west][inner sep=0.75pt]    {$\xi ^{\mu } =\frac{dx^{\mu }}{d\lambda }$};
            % Text Node
            \draw (251,173.4) node [anchor=north west][inner sep=0.75pt]    {$\xi ^{\alpha } \nabla _{\alpha } \xi ^{\mu } =0$};
            % Text Node
            \draw (70,202.4) node [anchor=north west][inner sep=0.75pt]    {$\Sigma $};
            
            
            \end{tikzpicture}
            
            \caption{A null hypersurface $\Sigma$ is generated by null geodesics where we have chosen $\lambda$ to be affine.}
            \end{center}

            \end{figure}
	
	The null hypersurface can be thought of being made up of these null generators. 
	
		We need some other concepts for hypersurfaces, two things we might want to know how how we can induce a metric on a hyper surface and how we can induce a volume on a hypersurface. 
		
	\section{Induced metric}
In the past by restricting to curves we have in a sense considered the reduced metric on some hypersurface $\Sigma$. This procedure will generally work, but we would like to examine this more generally. 
	
	
	\begin{figure}[h]
	\begin{center}
	

\tikzset{every picture/.style={line width=0.75pt}} %set default line width to 0.75pt        

\begin{tikzpicture}[x=0.75pt,y=0.75pt,yscale=-1,xscale=1]
%uncomment if require: \path (0,300); %set diagram left start at 0, and has height of 300

%Straight Lines [id:da5750335369028299] 
\draw    (146.5,198) -- (146.5,41) ;
\draw [shift={(146.5,39)}, rotate = 450] [color={rgb, 255:red, 0; green, 0; blue, 0 }  ][line width=0.75]    (10.93,-3.29) .. controls (6.95,-1.4) and (3.31,-0.3) .. (0,0) .. controls (3.31,0.3) and (6.95,1.4) .. (10.93,3.29)   ;
%Straight Lines [id:da9759337438083686] 
\draw    (130.5,185) -- (309.5,185) ;
\draw [shift={(311.5,185)}, rotate = 180] [color={rgb, 255:red, 0; green, 0; blue, 0 }  ][line width=0.75]    (10.93,-3.29) .. controls (6.95,-1.4) and (3.31,-0.3) .. (0,0) .. controls (3.31,0.3) and (6.95,1.4) .. (10.93,3.29)   ;
%Straight Lines [id:da49477524457475197] 
\draw    (159.5,172) -- (83.91,247.59) ;
\draw [shift={(82.5,249)}, rotate = 315] [color={rgb, 255:red, 0; green, 0; blue, 0 }  ][line width=0.75]    (10.93,-3.29) .. controls (6.95,-1.4) and (3.31,-0.3) .. (0,0) .. controls (3.31,0.3) and (6.95,1.4) .. (10.93,3.29)   ;
%Straight Lines [id:da6894225500593614] 
\draw    (86.5,141) -- (131.5,96) ;
%Straight Lines [id:da2334432220166296] 
\draw    (279.5,157) -- (324.5,112) ;
%Curve Lines [id:da8018686809959821] 
\draw    (86.5,141) .. controls (126.5,111) and (239.5,187) .. (279.5,157) ;
%Curve Lines [id:da3117827831327262] 
\draw    (131.5,96) .. controls (171.5,66) and (289.5,118) .. (324.5,112) ;
%Straight Lines [id:da7373284250910577] 
\draw [line width=0.75]    (168,93) -- (190.47,55.71) ;
\draw [shift={(191.5,54)}, rotate = 481.07] [color={rgb, 255:red, 0; green, 0; blue, 0 }  ][line width=0.75]    (10.93,-3.29) .. controls (6.95,-1.4) and (3.31,-0.3) .. (0,0) .. controls (3.31,0.3) and (6.95,1.4) .. (10.93,3.29)   ;
%Straight Lines [id:da9153602483642644] 
\draw [line width=0.75]    (205,101) -- (215.93,63.92) ;
\draw [shift={(216.5,62)}, rotate = 466.43] [color={rgb, 255:red, 0; green, 0; blue, 0 }  ][line width=0.75]    (10.93,-3.29) .. controls (6.95,-1.4) and (3.31,-0.3) .. (0,0) .. controls (3.31,0.3) and (6.95,1.4) .. (10.93,3.29)   ;
%Straight Lines [id:da23950322331304008] 
\draw [line width=0.75]    (284,119) -- (284.48,72) ;
\draw [shift={(284.5,70)}, rotate = 450.58] [color={rgb, 255:red, 0; green, 0; blue, 0 }  ][line width=0.75]    (10.93,-3.29) .. controls (6.95,-1.4) and (3.31,-0.3) .. (0,0) .. controls (3.31,0.3) and (6.95,1.4) .. (10.93,3.29)   ;
%Straight Lines [id:da36003150796097483] 
\draw [line width=0.75]    (239,107) -- (244.23,67.98) ;
\draw [shift={(244.5,66)}, rotate = 457.64] [color={rgb, 255:red, 0; green, 0; blue, 0 }  ][line width=0.75]    (10.93,-3.29) .. controls (6.95,-1.4) and (3.31,-0.3) .. (0,0) .. controls (3.31,0.3) and (6.95,1.4) .. (10.93,3.29)   ;
%Curve Lines [id:da14167491362159934] 
\draw    (316.5,36) .. controls (299.93,23.32) and (275.75,25.86) .. (264.35,58.43) ;
\draw [shift={(263.5,61)}, rotate = 287.45] [fill={rgb, 255:red, 0; green, 0; blue, 0 }  ][line width=0.08]  [draw opacity=0] (8.93,-4.29) -- (0,0) -- (8.93,4.29) -- cycle    ;
%Straight Lines [id:da011932319959848892] 
\draw    (302.5,119) -- (315.5,120) ;
%Straight Lines [id:da49804044265785086] 
\draw    (302.5,119) -- (310.5,112) ;
%Straight Lines [id:da27603099786895346] 
\draw [line width=0.75]    (304,139) -- (304.48,92) ;
\draw [shift={(304.5,90)}, rotate = 450.58] [color={rgb, 255:red, 0; green, 0; blue, 0 }  ][line width=0.75]    (10.93,-3.29) .. controls (6.95,-1.4) and (3.31,-0.3) .. (0,0) .. controls (3.31,0.3) and (6.95,1.4) .. (10.93,3.29)   ;
%Straight Lines [id:da06953426972386123] 
\draw [line width=0.75]    (259,127) -- (264.23,87.98) ;
\draw [shift={(264.5,86)}, rotate = 457.64] [color={rgb, 255:red, 0; green, 0; blue, 0 }  ][line width=0.75]    (10.93,-3.29) .. controls (6.95,-1.4) and (3.31,-0.3) .. (0,0) .. controls (3.31,0.3) and (6.95,1.4) .. (10.93,3.29)   ;

% Text Node
\draw (90,83.4) node [anchor=north west][inner sep=0.75pt]    {$\mathcal{M}$};
% Text Node
\draw (319,34.4) node [anchor=north west][inner sep=0.75pt]    {$n^{\mu }$};
% Text Node
\draw (341,36) node [anchor=north west][inner sep=0.75pt]   [align=left] {\mbox{-} unit normal};
% Text Node
\draw (189,226.4) node [anchor=north west][inner sep=0.75pt]    {$g_{\mu \nu }$};
% Text Node
\draw (283,149.4) node [anchor=north west][inner sep=0.75pt]    {$f=f_{*}$};
% Text Node
\draw (119,115.4) node [anchor=north west][inner sep=0.75pt]    {$\Sigma $};


\end{tikzpicture}
	\caption{FIGURE 2}
	\end{center}
	\end{figure}
	
	
	Formally, we define the coordinates on $\Sigma: y_i,  i = 1,2,3$ for spatial indices ($i =0,1,2$ if we would like to include timelike entries). More generally $i$ will run over $(n-1)$ of an $n$ dimensional manifold's indices. Since $\Sigma$ is a submanifold there is a mapping $\phi: y^i \to x^\mu$ from $\Sigma$ to $\mathcal{M}$. Then the pullback of the metric $g_{\mu \nu}$ will naturally induce a metric by the mapping $\phi$  to give a metric $\gamma_{ij}$ for $\Sigma$. This $\gamma_{ij}$ will be the induced metric on $\Sigma$. 
	The pullback acts on a metric to reduce it to a metric on the lower dimensional submanifold 
	\begin{align}
	\gamma _{ij} = \left( \phi^* g\right)_{ij} = \frac{\partial x^\mu}{\partial y^i} \frac{\partial x^\nu}{\partial y^j} g_{\mu \nu}.
	\end{align}

\subsection{Adapted Coordinates}
	In practice, if we use adapted coordinates defined by $x^\mu = (f, y^i)$ (notice that $\frac{\partial x^\mu}{\partial y^i} |_\Sigma$ is simple since $\frac{\partial f}{\partial y^i} |_\Sigma = 0$.) The figure below illustrates $\frac{\partial x^\mu}{\partial y^i}$  having 3 rows and 3 columns, which when acting on $g_{\mu \nu}$ will reduce the metric to only having 9 entries rather than 16. 	
	\begin{figure}[h]
	\begin{center}
	

\tikzset{every picture/.style={line width=0.75pt}} %set default line width to 0.75pt        

\begin{tikzpicture}[x=0.75pt,y=0.75pt,yscale=-1,xscale=1]
%uncomment if require: \path (0,300); %set diagram left start at 0, and has height of 300

%Straight Lines [id:da12366776773175481] 
\draw    (117.5,150) -- (174.5,150) ;
\draw [shift={(177.5,150)}, rotate = 180] [fill={rgb, 255:red, 0; green, 0; blue, 0 }  ][line width=0.08]  [draw opacity=0] (8.93,-4.29) -- (0,0) -- (8.93,4.29) -- cycle    ;
%Straight Lines [id:da35358256677310296] 
\draw    (211.5,110) -- (211.5,142) ;
\draw [shift={(211.5,145)}, rotate = 270] [fill={rgb, 255:red, 0; green, 0; blue, 0 }  ][line width=0.08]  [draw opacity=0] (8.93,-4.29) -- (0,0) -- (8.93,4.29) -- cycle    ;

% Text Node
\draw (68,108.4) node [anchor=north west][inner sep=0.75pt]  [font=\normalsize]  {${\textstyle \frac{\partial x^{\mu }}{\partial y^{i}} =}$};
% Text Node
\draw (113,99.4) node [anchor=north west][inner sep=0.75pt]  [font=\huge]  {$($};
% Text Node
\draw (170,98.4) node [anchor=north west][inner sep=0.75pt]  [font=\huge]  {$)$};
% Text Node
\draw (181,130.4) node [anchor=north west][inner sep=0.75pt]  [font=\small]  {$ \begin{array}{l}
i\\
\end{array}$};
% Text Node
\draw (190,98.4) node [anchor=north west][inner sep=0.75pt]    {$\mu $};
% Text Node
\draw (112,159) node [anchor=north west][inner sep=0.75pt]   [align=left] {4 columns};
% Text Node
\draw (224,115) node [anchor=north west][inner sep=0.75pt]   [align=left] {3 rows};


\end{tikzpicture}
	\caption{Adapted coordinate transform take $g_{\mu \nu} \to \gamma_{ij}$}
	\end{center}
	\end{figure}
	
	These adapted coordinates naturally pick out the $ij$ components of the metric since they are hit with the identity matrix, while the rest of the matrix is zero. The 0th index defines the hypersurface we are working on. This means that we can pick out pieces of the metric by choosing our adapted coordinates appropriately. 
	
	
	Often we also see the following approach. Define 
	\begin{align}
	\gamma_{\mu \nu} = g_{\mu \nu} - \sigma n_\mu n_\nu,
	\end{align}
	 where $n_\mu$ is a unit normal and $\sigma$ gives us the sign, $\sigma n_\mu n^\mu = \pm 1$. 
	
	eg, flat space cartesian coordinates and let $\Sigma$ be $t=0$, then $n^\mu$ TYPE OUT PICTURE 4 gives back what we want with zero in time component and in spatial components gives 3d flat space metric. 
	
	So $\gamma ^\mu _\nu$ is a projection operation -- when applied to a vector, it projects the vector into $\Sigma$. This works in the following way 
	\begin{align*}
	\gamma ^\mu _\nu &= \delta ^\mu _\nu -\sigma n^\mu n_\nu \\
	\gamma ^\mu _\nu n^\nu &= \delta ^\mu _\nu n^\nu - \sigma n^\mu (n_\nu n^\nu)\\
	 &= \ n^\mu - \sigma n^\mu (\sigma) \quad \quad (\sigma^2 = 1) \\
	&= n^\mu -n^\mu \\&= 0.
	\end{align*}
	This means when when $\gamma ^\mu _\nu$ is applied to a vector, $\gamma ^\mu _\nu V^\nu$, the action deletes all parts along $n^\mu$, leaving us with a vector that exists solely in $\Sigma$. 
	
	Similarly $\gamma _\mu ^\nu$ projects covariant indices into $\Sigma$. For $\hat T _{\alpha \beta}$ living in $\Sigma,$ we have 
	\begin{align}
	\hat T _{\alpha \beta} = \gamma_\alpha^\mu \gamma_\beta^\nu T_{\mu \nu} .
	\end{align}  
	Once evaluated  $T_{\mu \nu}  \to \hat T _{\alpha \beta}|_\Sigma$ is a tensor in $\Sigma$. 
	
	We can now also define a covariant derivative of the fully projected object as 
	\begin{align}
	D_\alpha \hat T_{\beta \gamma} = \gamma_{\alpha}^\mu  \gamma_{\beta}^\nu   \gamma_{\gamma}^\rho \left(\nabla_\mu T_{\nu \rho}\right) 
	\end{align}
	covariant derivative as defined by the induced metric $\gamma_{\mu \nu}$ in $\Sigma$.
	
	
	PICTURE 5 -- interested in splitting the manifold and then see how gamma varies with each slice. Note that in adapted coordinates $x^\mu = (f, y^i)$ then the tangent vector with one index down $n_\mu = (-\alpha, 0 ,0,0)$ and $\gamma_{ij} = g_{ij}$ basis of the 3+1 split and the ADM formulation -- this is the starting point for that. 
	
	\subsection{Gaussian Normal Coordinates}
	A particularly useful set of adapted coordinates are called Gaussian normal coordinates.. start with hypersurface $\Sigma$,  and give it coordinates along with a set of tangents $n$. Then starting at every point w/ coordinate $y^i$ shoot a geodesic where $n$ is the tangent to that geodesic, define a set of coordinates in the manifold around $\Sigma$ using this construction of shooting geodesics. Along each geodesic we choose the coordinates $x^\mu = (z,y^i)$, where the new coordinate $z$ is the affine parameter along the geodesic normalized that $z=0$ on $\Sigma$. This defines coordinates all around $\Sigma$. 
	\begin{figure}[h]
	\begin{center}
	

\tikzset{every picture/.style={line width=0.75pt}} %set default line width to 0.75pt        

\begin{tikzpicture}[x=0.75pt,y=0.75pt,yscale=-1,xscale=1]
%uncomment if require: \path (0,300); %set diagram left start at 0, and has height of 300

%Shape: Parallelogram [id:dp6552355992754251] 
\draw   (93.65,132) -- (244.5,132) -- (179.85,233) -- (29,233) -- cycle ;
%Straight Lines [id:da9293611837639084] 
\draw [line width=1.5]    (151,147) -- (151.47,101) ;
\draw [shift={(151.5,98)}, rotate = 450.58] [color={rgb, 255:red, 0; green, 0; blue, 0 }  ][line width=1.5]    (14.21,-4.28) .. controls (9.04,-1.82) and (4.3,-0.39) .. (0,0) .. controls (4.3,0.39) and (9.04,1.82) .. (14.21,4.28)   ;
%Straight Lines [id:da014956757076197036] 
\draw [line width=1.5]    (105,142) -- (110.1,103.97) ;
\draw [shift={(110.5,101)}, rotate = 457.64] [color={rgb, 255:red, 0; green, 0; blue, 0 }  ][line width=1.5]    (14.21,-4.28) .. controls (9.04,-1.82) and (4.3,-0.39) .. (0,0) .. controls (4.3,0.39) and (9.04,1.82) .. (14.21,4.28)   ;
%Straight Lines [id:da5527883605394042] 
\draw [line width=1.5]    (188,148) -- (188.47,102) ;
\draw [shift={(188.5,99)}, rotate = 450.58] [color={rgb, 255:red, 0; green, 0; blue, 0 }  ][line width=1.5]    (14.21,-4.28) .. controls (9.04,-1.82) and (4.3,-0.39) .. (0,0) .. controls (4.3,0.39) and (9.04,1.82) .. (14.21,4.28)   ;
%Straight Lines [id:da8352439185862821] 
\draw [line width=0.75]    (150.65,148.06) -- (153.84,50.15) ;
\draw [shift={(153.9,48.15)}, rotate = 451.87] [color={rgb, 255:red, 0; green, 0; blue, 0 }  ][line width=0.75]    (10.93,-3.29) .. controls (6.95,-1.4) and (3.31,-0.3) .. (0,0) .. controls (3.31,0.3) and (6.95,1.4) .. (10.93,3.29)   ;
%Straight Lines [id:da6000791067306068] 
\draw [line width=0.75]    (106.95,136.59) -- (114.57,55.13) ;
\draw [shift={(114.75,53.14)}, rotate = 455.34] [color={rgb, 255:red, 0; green, 0; blue, 0 }  ][line width=0.75]    (10.93,-3.29) .. controls (6.95,-1.4) and (3.31,-0.3) .. (0,0) .. controls (3.31,0.3) and (6.95,1.4) .. (10.93,3.29)   ;
%Straight Lines [id:da8850421527722161] 
\draw [line width=0.75]    (187.58,149.12) -- (190.77,51.21) ;
\draw [shift={(190.83,49.21)}, rotate = 451.87] [color={rgb, 255:red, 0; green, 0; blue, 0 }  ][line width=0.75]    (10.93,-3.29) .. controls (6.95,-1.4) and (3.31,-0.3) .. (0,0) .. controls (3.31,0.3) and (6.95,1.4) .. (10.93,3.29)   ;
%Curve Lines [id:da6914992725706202] 
\draw    (275.5,41) .. controls (277.45,26.37) and (226.16,21.26) .. (201.35,48.82) ;
\draw [shift={(199.5,51)}, rotate = 308.65999999999997] [fill={rgb, 255:red, 0; green, 0; blue, 0 }  ][line width=0.08]  [draw opacity=0] (8.93,-4.29) -- (0,0) -- (8.93,4.29) -- cycle    ;
%Straight Lines [id:da6675846270176469] 
\draw    (71.5,200) -- (61.5,218) ;
%Straight Lines [id:da6037672459919492] 
\draw    (79.5,217) -- (58.5,211) ;
%Shape: Rectangle [id:dp7204797967851337] 
\draw   (405,73) -- (523.5,73) -- (523.5,140.71) -- (405,140.71) -- cycle ;
%Straight Lines [id:da005059170905314048] 
\draw    (418,108) -- (508.5,108) ;
%Straight Lines [id:da10014975226116496] 
\draw    (464.25,88.86) -- (464.25,124.86) ;
%Curve Lines [id:da5422772912406553] 
\draw    (427.5,123) .. controls (418.5,115) and (420.5,97) .. (431,90) ;
%Curve Lines [id:da5149721866037893] 
\draw    (496.5,124) .. controls (509.5,114) and (504.5,98) .. (500,91) ;
%Curve Lines [id:da6027822575121178] 
\draw    (447.5,123) .. controls (444.5,115) and (445.5,97) .. (450,90) ;
%Curve Lines [id:da615059458332172] 
\draw    (482.5,123) .. controls (489.5,114) and (488.5,102) .. (482,91) ;
%Curve Lines [id:da3931192613829648] 
\draw    (334.5,55) .. controls (408.75,99.55) and (210.53,119.6) .. (250.22,194.71) ;
\draw [shift={(251.5,197)}, rotate = 239.7] [fill={rgb, 255:red, 0; green, 0; blue, 0 }  ][line width=0.08]  [draw opacity=0] (8.93,-4.29) -- (0,0) -- (8.93,4.29) -- cycle    ;

% Text Node
\draw (193,224.4) node [anchor=north west][inner sep=0.75pt]  [font=\Large]  {$\Sigma $};
% Text Node
\draw (216,102.4) node [anchor=north west][inner sep=0.75pt]    {$n^{\mu }$};
% Text Node
\draw (270,43) node [anchor=north west][inner sep=0.75pt]   [align=left] {geodesic};
% Text Node
\draw (87,210.4) node [anchor=north west][inner sep=0.75pt]  [font=\scriptsize]  {$y^{i}$};
% Text Node
\draw (191,142.4) node [anchor=north west][inner sep=0.75pt]    {$y^{i}$};
% Text Node
\draw (372,59.4) node [anchor=north west][inner sep=0.75pt]  [font=\large]  {$\mathcal{M}$};
% Text Node
\draw (253,191) node [anchor=north west][inner sep=0.75pt]   [align=left] {coordinates};
% Text Node
\draw (347,188.4) node [anchor=north west][inner sep=0.75pt]    {$x^{\mu } =\left( z,y^{i}\right)$};


\end{tikzpicture}

\end{center}
\caption{Gaussian Normal Coordinates}
\end{figure}
	
		However, we might have a problem far away from $\Sigma$ since the geodesics are not necessarily parallel to one-another and will eventually intersect. The means that Gaussian normal coordinates will only work in some neighborhood around $\Sigma$, where the size of the neighborhood depends on $\Sigma$. 
			
	With Gaussian normal coordinates  coordinates one can show that the metric takes the form $ds^2 = \sigma dz^2 + \gamma_{ij}dx^idx^j$. where $\sigma = \pm 1 = n_\alpha n^\alpha$. note that no cross terms $g_0i$ use up \textbf{all} of our coordinate freedom to fix four of the metric components. $g_{0\mu}$ are fixed. 
	From this we can read off $\gamma_{ij}$ easily as   $\gamma_{ij}|_{z=0} = (\phi^* g)_{ij}$.
	
\section{Induced Volume}
	In addition to thinking about the induced metric on some hypersurface $\Sigma$ we would also like to think about the induced volume element. With an induced volume element we can do integrals on $\Sigma$. 
	
	$\Sigma$ has a natural volume element UNDERTILDE ON EPSILON and all coordinate basis elements  $\hat{\epsilon} = \sqrt{|\gamma|} dy^1\wedge dy^2 \wedge ... \wedge dy^{n-1}$ before had  $\epsilon = \sqrt{|g|} dx^1\wedge dx^2 \wedge ... \wedge dx^{n-1}$
	
	The two are related, in coordinates we have that  $\hat{\epsilon}_{\mu_1...\mu_n-1} = n^\alpha \epsilon{\alpha \mu_1...\mu_n-1} .$
	
	This will prove useful for Stoke's theorem. Stoke's theorem is how we relate divergence of vector field to the surface integral over that vector field -- this is a very fundamental statement in the language of differential forms. 
	
	Differential forms -- $\int_U d\omega $
	
	WRITE OUT WHAT IS IN PICTURE 7
	
	From this and with the Hodge dual (*) you can derive the expression $\int_U d^n x \sqrt{|g|} \nabla_\mu V^\mu = \int_{\partial U} d^{n-1} y \sqrt{|\gamma|} n_mu V^\mu$. 

	\begin{figure}[h]
	\begin{center}
	

\tikzset{every picture/.style={line width=0.75pt}} %set default line width to 0.75pt        

\begin{tikzpicture}[x=0.75pt,y=0.75pt,yscale=-1,xscale=1]
%uncomment if require: \path (0,300); %set diagram left start at 0, and has height of 300

%Shape: Circle [id:dp2638096248143411] 
\draw  [line width=1.5]  (187,112.75) .. controls (187,68.15) and (223.15,32) .. (267.75,32) .. controls (312.35,32) and (348.5,68.15) .. (348.5,112.75) .. controls (348.5,157.35) and (312.35,193.5) .. (267.75,193.5) .. controls (223.15,193.5) and (187,157.35) .. (187,112.75) -- cycle ;
%Curve Lines [id:da34589449985318255] 
\draw    (332.5,98) .. controls (372.5,68) and (344,123) .. (384,93) ;
%Curve Lines [id:da7403506187185493] 
\draw    (333.5,68) .. controls (355.5,50) and (355.5,50) .. (369.5,37) ;
%Straight Lines [id:da24899961421737582] 
\draw    (267.75,32) -- (267.75,6) ;
\draw [shift={(267.75,4)}, rotate = 450] [color={rgb, 255:red, 0; green, 0; blue, 0 }  ][line width=0.75]    (10.93,-3.29) .. controls (6.95,-1.4) and (3.31,-0.3) .. (0,0) .. controls (3.31,0.3) and (6.95,1.4) .. (10.93,3.29)   ;
%Straight Lines [id:da4341330060057167] 
\draw    (314.75,46) -- (332.36,20.64) ;
\draw [shift={(333.5,19)}, rotate = 484.78] [color={rgb, 255:red, 0; green, 0; blue, 0 }  ][line width=0.75]    (10.93,-3.29) .. controls (6.95,-1.4) and (3.31,-0.3) .. (0,0) .. controls (3.31,0.3) and (6.95,1.4) .. (10.93,3.29)   ;
%Straight Lines [id:da1576244169008274] 
\draw    (213.75,52) -- (195.72,28.58) ;
\draw [shift={(194.5,27)}, rotate = 412.4] [color={rgb, 255:red, 0; green, 0; blue, 0 }  ][line width=0.75]    (10.93,-3.29) .. controls (6.95,-1.4) and (3.31,-0.3) .. (0,0) .. controls (3.31,0.3) and (6.95,1.4) .. (10.93,3.29)   ;
%Curve Lines [id:da22639088718445777] 
\draw    (187,112.75) .. controls (204.5,136) and (308.5,149) .. (348.5,112.75) ;
%Straight Lines [id:da5860359853814017] 
\draw    (207.75,168) -- (189.63,194.35) ;
\draw [shift={(188.5,196)}, rotate = 304.51] [color={rgb, 255:red, 0; green, 0; blue, 0 }  ][line width=0.75]    (10.93,-3.29) .. controls (6.95,-1.4) and (3.31,-0.3) .. (0,0) .. controls (3.31,0.3) and (6.95,1.4) .. (10.93,3.29)   ;
%Straight Lines [id:da4948700601667386] 
\draw    (267.75,193.5) -- (267.75,221) ;
\draw [shift={(267.75,223)}, rotate = 270] [color={rgb, 255:red, 0; green, 0; blue, 0 }  ][line width=0.75]    (10.93,-3.29) .. controls (6.95,-1.4) and (3.31,-0.3) .. (0,0) .. controls (3.31,0.3) and (6.95,1.4) .. (10.93,3.29)   ;
%Straight Lines [id:da504827644253663] 
\draw    (322.75,172.5) -- (340.37,198.35) ;
\draw [shift={(341.5,200)}, rotate = 235.71] [color={rgb, 255:red, 0; green, 0; blue, 0 }  ][line width=0.75]    (10.93,-3.29) .. controls (6.95,-1.4) and (3.31,-0.3) .. (0,0) .. controls (3.31,0.3) and (6.95,1.4) .. (10.93,3.29)   ;
%Straight Lines [id:da7489986877364485] 
\draw    (290.75,134.5) -- (308.37,160.35) ;
\draw [shift={(309.5,162)}, rotate = 235.71] [color={rgb, 255:red, 0; green, 0; blue, 0 }  ][line width=0.75]    (10.93,-3.29) .. controls (6.95,-1.4) and (3.31,-0.3) .. (0,0) .. controls (3.31,0.3) and (6.95,1.4) .. (10.93,3.29)   ;
%Straight Lines [id:da5362765946948465] 
\draw    (247.75,134.5) -- (236.22,164.14) ;
\draw [shift={(235.5,166)}, rotate = 291.25] [color={rgb, 255:red, 0; green, 0; blue, 0 }  ][line width=0.75]    (10.93,-3.29) .. controls (6.95,-1.4) and (3.31,-0.3) .. (0,0) .. controls (3.31,0.3) and (6.95,1.4) .. (10.93,3.29)   ;
%Straight Lines [id:da3975308732937963] 
\draw    (282.75,112.5) -- (301.41,83.68) ;
\draw [shift={(302.5,82)}, rotate = 482.92] [color={rgb, 255:red, 0; green, 0; blue, 0 }  ][line width=0.75]    (10.93,-3.29) .. controls (6.95,-1.4) and (3.31,-0.3) .. (0,0) .. controls (3.31,0.3) and (6.95,1.4) .. (10.93,3.29)   ;
%Straight Lines [id:da7581059693978076] 
\draw    (238.75,102.5) -- (238.52,72) ;
\draw [shift={(238.5,70)}, rotate = 449.56] [color={rgb, 255:red, 0; green, 0; blue, 0 }  ][line width=0.75]    (10.93,-3.29) .. controls (6.95,-1.4) and (3.31,-0.3) .. (0,0) .. controls (3.31,0.3) and (6.95,1.4) .. (10.93,3.29)   ;
%Straight Lines [id:da012284259265882058] 
\draw    (189.75,89) -- (158.42,79.58) ;
\draw [shift={(156.5,79)}, rotate = 376.74] [color={rgb, 255:red, 0; green, 0; blue, 0 }  ][line width=0.75]    (10.93,-3.29) .. controls (6.95,-1.4) and (3.31,-0.3) .. (0,0) .. controls (3.31,0.3) and (6.95,1.4) .. (10.93,3.29)   ;
%Curve Lines [id:da7983631911403086] 
\draw    (118.5,88) .. controls (145.5,63) and (135.5,108) .. (158,85) ;

% Text Node
\draw (390,119.4) node [anchor=north west][inner sep=0.75pt]  [font=\Large]  {$\int _{U} \nabla _{\mu } V^{\mu } \ \ \rightarrow \ \ \int _{\partial U} n^{\mu } V_{\mu }$};
% Text Node
\draw (373,27.4) node [anchor=north west][inner sep=0.75pt]    {$\partial U$};
% Text Node
\draw (386,82.4) node [anchor=north west][inner sep=0.75pt]    {$U$};
% Text Node
\draw (99,76.4) node [anchor=north west][inner sep=0.75pt]    {$n^{\mu }$};


\end{tikzpicture}
\end{center}
\caption{THINK GAUSS' LAW}
\end{figure}

Examples: Conserved Current -- $\nabla_\mu J^\mu = 0$ local conservation want to show how this gives global charge 

PICTURE 9

0 = $\int d^4 x \sqrt{-g} \nabla_\mu J^\mu = -\int_{\Sigma_2} d^3 y \sqrt{\gamma} n_\mu J^\mu +\int_{\Sigma_1} d^3 y \sqrt{\gamma} n_\mu J^\mu$ NOTE THAT WE'VE TAKEN A TIME SLIZE SO WE HAVE ONLY SPATIAL PART. ALSO BE CAFEFUL OF SIGN IN FRONT OF BOTH INTEGRAL MAKE COMMENT ABOUT HOW IT DEPENDS ON N AND THE CONVENTION WE HAVE IS FOR FUTURE POINT  look at minus part on picture 


Then $\int_{\Sigma_2} d^3 y \sqrt{\gamma} n_\mu J^\mu=\int_{\Sigma_1} d^3 y \sqrt{\gamma} n_\mu J^\mu$ --> conserved charge

to interpret this go to flat space , take $t$= constant surfaces $\implies$ $n^\mu = (1,0,0,0)$ and $n_\mu = (-1,0,0,0)$ $\sqrt{\gamma} = 1$ and $n_\mu J^\mu = -J^t = -\rho$ (charge density)

$\int_\Sigma d^3 y \sqrt{|\gamma|} n_\mu J^\mu = \int d^3 y (-\rho) = -Q$ which is the total charge 

We can then define the total charge associated with the current four vector in a curved spacetime  as 
\begin{align}
Q = -\int_\Sigma d^3 y \sqrt{|\gamma|} n_\mu J^\mu. 
\end{align}

We can further reduce the charge integral to an integral over a two surface 
Further use Maxwell's equations 
$\nabla_\nu F^{\mu \nu} = J^\mu$ -- plug this in to definition of Q
\begin{align}
Q = -\int_\Sigma d^3 y \sqrt{|\gamma|} n_\mu (\nabla_\nu F^{\mu \nu}).  
\end{align}

When $F^{\mu \nu}$ is a two form, the divergence piece is like the exterior derivative of the two form $dF$ which means we can apply Stoke's theorem again. 

Then we can reduce this to 
\begin{align}
Q = -\int_{\partial \Sigma} d^2 z \sqrt{|\gamma^{(2)}|} n_\mu s_\nu F^{\mu \nu}
\end{align}
where $s_\nu$ is a normal to the boundary. Note that everything here is now the 2d induced metric and the integral over a two dimensional surface. 

We've also now shown that the ocnservation of local current $\nabla_\mu J^\mu \implies Q_2 = Q_1$. This is like =Gauss' law in integral form which is relating the charge to a surface integral. 

\begin{figure}[h]
\begin{center}


\tikzset{every picture/.style={line width=0.75pt}} %set default line width to 0.75pt        

\begin{tikzpicture}[x=0.75pt,y=0.75pt,yscale=-1,xscale=1]
%uncomment if require: \path (0,300); %set diagram left start at 0, and has height of 300

%Shape: Rectangle [id:dp8405984036653336] 
\draw   (212.4,46.03) -- (592.5,46.03) -- (592.5,187.89) -- (212.4,187.89) -- cycle ;
%Shape: Rectangle [id:dp8944632707692322] 
\draw   (103,112.41) -- (483.1,112.41) -- (483.1,254.28) -- (103,254.28) -- cycle ;
%Straight Lines [id:da31071819751250995] 
\draw    (103,112.41) -- (212.4,46.03) ;
%Straight Lines [id:da2972255634712586] 
\draw    (103,254.28) -- (212.4,187.89) ;
%Straight Lines [id:da7944030176473238] 
\draw    (483.1,254.28) -- (592.5,187.89) ;
%Straight Lines [id:da17149662164659474] 
\draw    (483.1,112.41) -- (592.5,46.03) ;
%Straight Lines [id:da5007091736373377] 
\draw [line width=1.5]    (212.4,16) -- (212.4,187.89) ;
\draw [shift={(212.4,12)}, rotate = 90] [fill={rgb, 255:red, 0; green, 0; blue, 0 }  ][line width=0.08]  [draw opacity=0] (11.61,-5.58) -- (0,0) -- (11.61,5.58) -- cycle    ;
%Straight Lines [id:da9357219890351993] 
\draw [line width=1.5]    (90.92,261.92) -- (212.4,187.89) ;
\draw [shift={(87.5,264)}, rotate = 328.64] [fill={rgb, 255:red, 0; green, 0; blue, 0 }  ][line width=0.08]  [draw opacity=0] (11.61,-5.58) -- (0,0) -- (11.61,5.58) -- cycle    ;
%Straight Lines [id:da21248576616139125] 
\draw [line width=1.5]    (629.5,187.89) -- (212.4,187.89) ;
\draw [shift={(633.5,187.89)}, rotate = 180] [fill={rgb, 255:red, 0; green, 0; blue, 0 }  ][line width=0.08]  [draw opacity=0] (11.61,-5.58) -- (0,0) -- (11.61,5.58) -- cycle    ;
%Shape: Can [id:dp09968257335739428] 
\draw   (411.43,74.27) -- (411.56,220.62) .. controls (411.57,231.84) and (381.24,240.97) .. (343.83,241) .. controls (306.41,241.03) and (276.07,231.96) .. (276.06,220.73) -- (275.93,74.38) .. controls (275.92,63.16) and (306.24,54.03) .. (343.66,54) .. controls (381.08,53.97) and (411.42,63.04) .. (411.43,74.27) .. controls (411.44,85.49) and (381.11,94.62) .. (343.7,94.65) .. controls (306.28,94.68) and (275.94,85.61) .. (275.93,74.38) ;
%Curve Lines [id:da2942112519407789] 
\draw  [dash pattern={on 0.84pt off 2.51pt}]  (276.06,220.73) .. controls (308.5,181) and (389.53,191.96) .. (411.56,220.62) ;
%Curve Lines [id:da7388591007579068] 
\draw    (326.5,280) .. controls (319.5,255) and (392.5,277) .. (388.5,215) ;
%Curve Lines [id:da678163933514482] 
\draw    (50.5,171) .. controls (90.1,141.3) and (61.09,178.25) .. (99.32,149.88) ;
\draw [shift={(100.5,149)}, rotate = 503.13] [color={rgb, 255:red, 0; green, 0; blue, 0 }  ][line width=0.75]    (10.93,-3.29) .. controls (6.95,-1.4) and (3.31,-0.3) .. (0,0) .. controls (3.31,0.3) and (6.95,1.4) .. (10.93,3.29)   ;
%Curve Lines [id:da8675216440066522] 
\draw    (49.5,235) .. controls (64.2,235.98) and (70.26,241.76) .. (98.73,232.58) ;
\draw [shift={(100.5,232)}, rotate = 521.5699999999999] [color={rgb, 255:red, 0; green, 0; blue, 0 }  ][line width=0.75]    (10.93,-3.29) .. controls (6.95,-1.4) and (3.31,-0.3) .. (0,0) .. controls (3.31,0.3) and (6.95,1.4) .. (10.93,3.29)   ;
%Straight Lines [id:da944465362561882] 
\draw    (347,60) -- (347,17) ;
\draw [shift={(347,15)}, rotate = 450] [color={rgb, 255:red, 0; green, 0; blue, 0 }  ][line width=0.75]    (10.93,-3.29) .. controls (6.95,-1.4) and (3.31,-0.3) .. (0,0) .. controls (3.31,0.3) and (6.95,1.4) .. (10.93,3.29)   ;
%Straight Lines [id:da6162892024855098] 
\draw    (564,135) -- (632.5,135) ;
\draw [shift={(634.5,135)}, rotate = 180] [color={rgb, 255:red, 0; green, 0; blue, 0 }  ][line width=0.75]    (10.93,-3.29) .. controls (6.95,-1.4) and (3.31,-0.3) .. (0,0) .. controls (3.31,0.3) and (6.95,1.4) .. (10.93,3.29)   ;
%Straight Lines [id:da7510063017582482] 
\draw    (331,63) -- (331.48,102) ;
\draw [shift={(331.5,104)}, rotate = 269.3] [color={rgb, 255:red, 0; green, 0; blue, 0 }  ][line width=0.75]    (10.93,-3.29) .. controls (6.95,-1.4) and (3.31,-0.3) .. (0,0) .. controls (3.31,0.3) and (6.95,1.4) .. (10.93,3.29)   ;
%Curve Lines [id:da39323040547336063] 
\draw    (251.5,83) .. controls (269.5,45) and (286,106) .. (326,76) ;

% Text Node
\draw (333,269.4) node [anchor=north west][inner sep=0.75pt]    {$J^{\mu } \neq 0$};
% Text Node
\draw (585,208.4) node [anchor=north west][inner sep=0.75pt]  [font=\large]  {$ \begin{array}{l}
\Sigma _{1}\\
\end{array}$};
% Text Node
\draw (611,34.4) node [anchor=north west][inner sep=0.75pt]  [font=\large]  {$ \begin{array}{l}
\Sigma _{2}\\
\end{array}$};
% Text Node
\draw (168,11.4) node [anchor=north west][inner sep=0.75pt]  [font=\Large]  {$\mathcal{M}$};
% Text Node
\draw (31,171) node [anchor=north west][inner sep=0.75pt]   [align=left] {very\\far\\away};
% Text Node
\draw (54,53.4) node [anchor=north west][inner sep=0.75pt]    {$\nabla _{\mu } J^{\mu } =0$};
% Text Node
\draw (357,4.4) node [anchor=north west][inner sep=0.75pt]    {$n^{\mu }$};
% Text Node
\draw (625,110.4) node [anchor=north west][inner sep=0.75pt]    {$n^{\mu }$};
% Text Node
\draw (233,79.4) node [anchor=north west][inner sep=0.75pt]    {$-n^{\mu }$};


\end{tikzpicture}

\end{center}
\caption{OK}
\end{figure}

Calculation w/ appropriate$ F_{\mu \nu}$ from a point charge with charge qu in flat space and then we find $Q=q$. 
%%%%%%%%%%%%%%%%%%%%%%%%%%%%%%%%%%%%%%%%%%
\end{document}