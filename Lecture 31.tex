\documentclass[10pt]{article}
\usepackage{NotesTeX} %/Path/to/package should be replaced with package location
\usepackage{lipsum}
\usepackage{tensor}
\usepackage{amsmath,amsthm,amssymb}
\usepackage{hyperref}
\usepackage{tikz}
\usepackage{tikz-cd}
\usepackage{pgfplots}
\tikzcdset{every label/.append style = {font = \small}}
\tikzcdset{row sep/normal=3.5em}
\tikzcdset{column sep/normal=3.5em}
\usetikzlibrary{shapes.geometric, arrows}

\usetikzlibrary{matrix}
\usetikzlibrary{decorations.markings,calc,shapes}
\usetikzlibrary{decorations.pathmorphing}
\usetikzlibrary{positioning}
\usepgfplotslibrary{fillbetween}
\usepackage{graphicx}
\usepackage{empheq}
\usepackage{physics}
\usepackage{siunitx}
\usepackage{tensor}

\usepackage{multicol}

\newcommand{\bs}{\textbackslash}


\title{{\Huge General Relativity}\\{\Large{Class 31}}} %replace with class number
\author{Josephina Wright}

\emailAdd{jrwright@utexas.edu} %replace with your email
\begin{document}
    \maketitle
    \flushbottom
    \newpage
    \pagestyle{fancynotes}
    \part{Black Holes: Conformal Diagrams and Causality}

\newline  A black hole is set of events (an event defined as a point in the spacetime manifold) that can never communicate with asymptotic infinity. This means that regions in the manifold cannot communicate with regions that are arbitrarily far away. Communication is done by sending a causal curve, timelike or null, out to infinity. 
	The boundary of a black hole is called the event horizon.
              	\section{Conformal Transformations }\label{sec:class_style}
              		 A metric is conformally related to another metric if
              		 \begin{equation}
              		 \widetilde{g_{uv}}=\omega(x^u)^2g_(uv)
              		 \end{equation}
              		 Where \(\omega\) is a function of spacectime. \(\omega\) is essentially re-scaling the proper times and proper distances in the spacetime in a position dependent manner. The resulting \(\widetilde{g_{uv}}\) is a conformal transform of \(g_{uv}\).
              	\(\widetilde{g_{uv}}\) is oft4en viewed as unphysical, since \(g_{uv}\) solves Einstein's field equations for some sources, but 	\(\widetilde{g_{uv}}\) will not solve Einstein's field equations with those sources. 	\(\widetilde{g_{uv}}\) is not a physical metric, simply a tool that can be used to understand spacetime.
              		 For example, the Weyl tensor can be calculated with \(g_{uv}\) or 	\(\widetilde{g_{uv}}\), while both options are equal, they are noted by 
              		  \begin{equation}
              		\widetilde{C^u_{v\rho\sigma}}=C^u_{v\rho\sigma}
              		 \end{equation}
              		 Where the following is not equal to each other
              		   \begin{equation}
              		\widetilde{C_{uv\rho\sigma}}\neq{C_{uv\rho\sigma}}
              		 \end{equation}
              		 This is because the index is lowered using	\(\widetilde{g_{uv}}\) and \(g_{uv}\) respectively. In other words, 
              		  \begin{equation}
              	 \widetilde{g_{u\alpha}}\widetilde{C^\alpha_{v\rho\sigma}}\neq{g_{u\alpha}C^\alpha_{v\rho\sigma}}
              		 \end{equation}
              		  	Another fact that makes conformal transformations useful in understanding a metric that is physical is, if given a null vector \(k^u\) (where \(k^uk^v\widetilde{g_{uv}}=0\)) then \(k^u\) is null with respect to \(\widetilde{g_{uv}}\):
              		  		 \begin{equation}
              		 k^uk^v\widetilde{g_{uv}}=\omega(k^uk^vg_{ev})^2=0
              		 \end{equation}
              		 Conformal transformations conveniently preserve the light cone structure. If two events are related by a null trajectory in \({g_{uv}}\), then they are also connected by a null trajectory in 	\(\widetilde{g_{uv}}\). 
              		 
               	\section{Conformal Diagrams}\label{sec:class_style}
               A goal of conformal diagrams is to take a spacetime and move to another set of coordinates chosen specifically so that light cones in the $\widetilde{T}$ and $\widetilde{R}$ plane retain the same shape as in flat space, in $45^\circ$, and such that $\widetilde{T}$ and $\widetilde{R}$ range over finite values an $t\xrightarrow{}\pm\infty$ and $r\xrightarrow{}\pm\infty$. In the left hand side of the figure below, every point corresponds to a closed surface such as a sphere, or a point.
                                             \begin{figure}[!h]
               \centering
  \begin{tikzpicture}[x=0.75pt,y=0.75pt,yscale=-1,xscale=1]
  %uncomment if require: \path (0,163); %set diagram left start at 0, and has height of 163
  \tikzstyle{arrow} = [thick,->,>=stealth]
  %LEFT
 \draw [arrow] (-200,100) -- (-200,0);
  \draw[arrow]  (-200,100) -- (-100,100);
%RIGHT
 \draw [arrow] (0,100) -- (0,0);
  \draw [arrow] (0,50) -- (81.5,50);

%Light Cones
\draw (60,20) ellipse (10 and 2.5);
\draw (60,40) ellipse (10 and 2.5);
\draw (50,20)--(70,40)
\draw (70,20)--(50,40)



\draw (20,70) ellipse (10 and 2.5);
\draw (20,90) ellipse (10 and 2.5);
\draw (10,70)--(30,90)
\draw (30,70)--(10,90)
%Draw top arrow
\draw [arrow] (-120,-20) -- (-20,-20);
\draw [arrow] (-50,50)-- (-130,50);
  % Label the points
   \draw (-125,35)node [anchor=north west][inner sep=0.75pt]    {\small closed surface};
  
  \draw  (91,40)node [anchor=north west][inner sep=0.75pt]    {$\widetilde{R}$};
  \draw (-15,0) node [anchor=north west][inner sep=0.75pt]    {$ \widetilde{T}$};

 \draw  (-90,100)node [anchor=north west][inner sep=0.75pt]    {r};
  \draw (-215,0) node [anchor=north west][inner sep=0.75pt]    {t};

  \draw  (-200,-30)node [anchor=north west][inner sep=0.75pt]    {$(t,r,\theta,\phi)$};
  \draw  (0,-30)node [anchor=north west][inner sep=0.75pt]    {$(\widetilde{T},\widetilde{R},\theta,\phi)$};
  \filldraw [black] (-140,50) circle (2pt);
  \end{tikzpicture}
  \caption{Goal of Conformal Diagram }
  \label{fig:lineup}
              \end{figure}  
               
               
               	The goal is to draw a spacetime diagram of all the spacetime, plus infinity. For example, lets look at Minkowski Spacetime. First, the line element is given by
              \begin{equation}
                  ds^2=-dt^2+dr^2+r^2d\Omega^2
              \end{equation}
              Now the coordinates can be transformed to $\widetilde{T}$ and $\widetilde{R}$, with $0\leq \widetilde{R}<\pi$ and $-\pi<\widetilde{T}<\pi$, and where $|\widetilde{T}|+\widetilde{R}<\pi$. With these coordinates, the Minkowski line element becomes
              \begin{equation}
                  ds^2=\frac{1}{(cos{-\tilde{T}}+cos{\tilde{R})}^2}(d\tilde{T}+d\tilde{R}+sin(\tilde{R})^2d\omega^2)
              \end{equation}
              This is a conformal factor multiplied by a simple metric, so if the first term were to instead be written as $\omega$, a conformal factor, then
                            \begin{equation}
                                d\widetilde{s}^2=\omega^2ds^2=(-d\tilde{T}^2+d\tilde{R}^2+sin(\tilde{R})^2d\omega^2)
                            \end{equation}
                           The metric $d\widetilde{s}$ is a simpler metric that can now be drawn with a conformal diagram, with the restricted ranges of $|\widetilde{T}|+\widetilde{R}<\pi$. The  $\widetilde{T}=constant$ surfaces cna never be inside the light cone, while  $\widetilde{R}=constant$ surfaces are required to be inside a light cone.
                           
                              \begin{figure}[!h]
               \centering
  \begin{tikzpicture}[x=0.75pt,y=0.75pt,yscale=-1,xscale=1]
  %uncomment if require: \path (0,163); %set diagram left start at 0, and has height of 163
  \tikzstyle{arrow} = [thick,->,>=stealth]
  %Draw the legend on the side
 \draw [arrow] (-120,81.5) -- (-120,40);
  \draw [arrow]  (-120,81.5) -- (-80,81.5);
%Draw triangles
 \draw[dashed] (0,0) -- (0,163);
 \draw (0,0) -- (150,81.5);
  \draw (0,163) -- (150,81.5);
 %Draw Vertical curves
 \draw (0,0) .. controls (20,74) and (20,89) .. (0,163);
  \draw (0,0) .. controls (60,74) and (60,89) .. (0,163);
   \draw (0,0) .. controls (100,74) and (100,89) .. (0,163);
   %Draw Horizontal curves
     \draw (0,81.5) -- (150,81.5);
   \draw (0,40) .. controls (60,40) and (100,60) .. (150,81.5);
 \draw (0,123) .. controls (60,123) and (100,103) .. (150,81.5);
 
%Draw Light Cone

\draw (42,55) ellipse (10 and 2.5);
\draw (42,75) ellipse (10 and 2.5);
\draw (32,55)--(52,75);
\draw (52,55)--(32,75);

  % Label the points
  \draw  (-80,81.5)node [anchor=north west][inner sep=0.75pt]    {$\widetilde{R}$};
  \draw (-120,20) node [anchor=north west][inner sep=0.75pt]    {$ \widetilde{T}$};
  
  \draw (10,-14) node [anchor=north west][inner sep=0.75pt]    {$\widetilde{R}=0, \widetilde{T}=\pi$};
   \draw (10,170) node [anchor=north west][inner sep=0.75pt]    {$\widetilde{R}=0, \widetilde{T}=-\pi$};
    \draw (163,81.5) node [anchor=north west][inner sep=0.75pt]    {$\widetilde{R}=\pi, \widetilde{T}=0$};
% Draw the points
\filldraw [black] (0,0) circle (2pt);
% Draw the points
\filldraw [black] (0,163) circle (2pt);
% Draw the points
\filldraw [black] (150,81.5) circle (2pt);

  \end{tikzpicture}
  \caption{Conformal Diagram of Minkowski Space: Restricted by $|\widetilde{T}|+\widetilde{R}<\pi$ }
  \label{fig:lineup}
              \end{figure}  
        
    \newline	Steps to derive conformal diagram for Minkowski
               	(Can be found in Appedix H of Carroll)
               	\begin{enumerate}
                    \item use null coordinates 
                    \begin{equation}
                   \begin{align}
                        u=t-r && v=t+r\\
                        t=\frac{v+u}{2} &&r=\frac{v-u}{2} \\
                        -\infty>u>\infty && -\infty>v>\infty\\
                        && v\geq{u}
                        \end{align}
                    \end{equation}
                    Where the \(v\geq{u}\) condition comes from the fact that r must be positive. 
                    In these coordinates, the metric can be written as 
                    \begin{equation}
                        ds^2=\frac{-1}{2}(dudv+dvdu)^2+r^2\Omega^2
                    \end{equation}
                    Remembering that \(dudv+dvdu=2dudv\)
                    \item Compactify coordinates
                    \begin{equation}
                   \begin{align}
                        u=tan(\widetilde{U}) && v=tan(\widetilde{V})\\
                        -\frac{\pi}{2}<tan(\widetilde{U})<\frac{\pi}{2} && -\frac{\pi}{2}<tan(\widetilde{V})<\frac{\pi}{2}
                        \end{align}
                    \end{equation}
            Now the coordinate system is such that whole metric can be written in a finite range of coordinates.
                                \begin{equation}
                   \begin{align}
                        du=sec^2(\widetilde{U}d\widetilde{U})&&dv=sec^2(\widetilde{V}d\widetilde{V})\\
                        \end{align}
                    \end{equation}
                    \begin{equation}
                        r^2=(\frac{sin(\widetilde{V}-\widetilde{U}}{2cos(\widetilde{V}cos(\widetilde{U}))})^2
                    \end{equation}
            
               	Now
               	\begin{equation}
               	    ds^2=\frac{1}{4cos^2\widetilde{U}cos^2\widetilde{V}}(-4d\widetilde{U}d\widetilde{V}+sin^2(\widetilde{V}-\widetilde{U})d\Omega^2)
               	\end{equation}
               	
               	\item define \((\widetilde{T},\widetilde{R})\), \(\widetilde{R}=\widetilde{v}-\widetilde{R}\), and \(\widetilde{R}=\widetilde{V}+\widetilde{U}\)
               	Resulting in 
               	\begin{equation}
               	 \begin{align} ds^2=\frac{1}{(4cos\widetilde{U}cos\widetilde{V})^2}&&(-d\widetilde{T}^2+d\widetilde{R}^2+sin^2\widetilde{R}d\Omega^2)\\
               	 \newline\\
               	  0\leq{\widetilde{R}}<\pi && |\widetilde{T}|+\widetilde{R}<\pi
               	  \end{align}
               	\end{equation}
               	\item Consider the conformally rescaled metric, then draw the spacetime diagrams
               	\begin{equation}
               	   \begin{align} d\widetilde{S}^2=\omega^2ds^2 &&  \omega=cos\widetilde{T}+cos\widetilde{R}
               	   \end{align}
               	\end{equation}
               	\begin{equation}
               	    d\widetilde{S}=-d\widetilde{T}^2+d\widetilde{R}^2+sin^2(\widetilde{R})d\Omega^2
               	\end{equation}

             \end{enumerate}
             We can now draw the allowed portions such a spacetime, remembering the constraint $|\widetilde{T}|+\widetilde{R}<\pi$, which confines the diagram to the non shaded region in the diagram. This cylinder can be 'unrolled' , with only the allowed portions for Minkowski Space, resulting in the conformal diagram from Figure 3. In these conformal diagrams, it is important to remember that the boundaries of the diagram and the points are not included in the space.
             \begin{figure}[!h]
                 \centering
           
                 \begin{tikzpicture}[x=0.75pt,y=0.75pt,yscale=-1,xscale=1]
                  \tikzstyle{arrow} = [thick,->,>=stealth]
                 %Draw cylinder
 \filldraw [color=gray!70, fill=gray!20, very thick](-20,20) ellipse (50 and 10);
\filldraw [color=gray!70, fill=gray!20, very thick](-20,120) ellipse (50 and 10);
\draw (-70,20)--(-70,120);
\draw (30,20)--(30,120);

\draw [dashed] (-25,30)--(-25,130);
\draw [dashed] (-15,10)--(-15,110);

%Draw conformal Diagram
\draw (55,70)--(100,30);
\draw (55,70)--(100,110);
\draw[dashed] (100,30)--(100,110);
\draw (100,30)--(145,70);
\draw (100,110)--(145,70);
\draw [arrow] (145,50)--(125,65);

% Draw the points
\filldraw [black] (-70,50)circle (1pt) ;
\filldraw [black] (-15,40)circle(1pt);
\filldraw [black] (30,50)circle (1pt);
\filldraw [black] (-25,60)circle(1pt) ;

\filldraw [black] (-70,90) circle(1pt);
\filldraw [black] (-15,80)circle(1pt);
\filldraw [black] (30,90)circle(1pt);
\filldraw [black] (25,50)circle(1pt);
\filldraw [black] (120,70)circle(1pt);
%Draw arrow to Conformal Diagram
\draw [arrow](30,50) .. controls (45,35) and (55,35) .. (75,50);
%Draw diamonds
\draw  (-70,50)--(-15,40) ;
\draw (-15,40)--(30,50) ;
\draw (30,50)--(-25,60) ;
\draw (-25,60)-- (-70,50);

\draw  (-70,90) -- (-15,80);
\draw  (-15,80)-- (30,90);
\draw (30,90)--(-25,100) ;
\draw  (-25,100)--(-70,90) ;

%make shaded region
    \draw [gray] (56,70)--(56,72);
            \draw [gray] (58,68)--(58,74);
                \draw [gray] (60,66)--(60,74);
                    \draw [gray] (62,62)--(62,76);
                    \draw [gray] (64,61)--(64,78);
                      \draw [gray] (66,61)--(66,80);
                    \draw [gray] (68,58)--(68,82);
                    \draw [gray] (70,58)--(70,84);
                  
                    \draw [gray] (72,56)--(72,84);
                    \draw [gray] (74,54)--(74,86);
                      \draw [gray] (76,52)--(76,88);
                    \draw [gray] (78,50)--(78,90);
                    \draw [gray] (80,48)--(80,92);
                                        \draw [gray] (82,46)--(82,94);
                    \draw [gray] (84,44)--(84,96);
                      \draw [gray] (86,42)--(86,98);
                    \draw [gray] (88,40)--(88,99);
                     \draw [gray] (90,38)--(90,100);
                                        \draw [gray] (92,36)--(92,102);
                    \draw [gray] (94,36)--(94,104);
                      \draw [gray] (96,34)--(96,106);
                    \draw [gray] (98,32)--(98,108);
  \draw  (145,40)node [anchor=north west][inner sep=0.75pt]    {$S^2}$};
    \draw  (-90,120)node [anchor=north west][inner sep=0.75pt]    {$S^3}$};

\end{tikzpicture}
                 \caption{Conformal Diagram Creation: from $S^3$ to $S^2$ space}
                 \label{fig:my_label}
             \end{figure}
             
  \section{Conformal Diagrams and Infinities}\label{sec:class_style}
  Light cones are at $45^\circ$ on this diagram, at the expense of other coordinates being bent in an odd way. This spacetime diagram is drawn on the conformally transformed metric, and the three points on the diagram are not points of the Minkowski Metric, rather they are different aspects of infinity. These points are placed on the diagram so that it becomes a manifold with a boundary.\newline This boundary contains five separate components, as labelled in figure 4. These are the infinities you would reach if you stayed at a fixed $\widetilde{R}$ in Minkowski Space. However, a curve at constant time would eventually reach spacelike infinity $i^0$. The last two components are $  \mathscr{I^+}$ and $  \mathscr{I^-}$. These are the solid boundaries that complete the triangle of the conformal diagram, and represent future and past null infinity respectively. All future directed light rays arrive at $  \mathscr{I^+}$, and all past directed light rays at $  \mathscr{I^-}$. In other words, $  \mathscr{I^-}$is where all null rays come from as they enter into the spacetime, and  $  \mathscr{I^+}$ is where they arrive in the future. \newline Illustrated in blue is a representation of the trajectory of an observer on a surface of very large $\widetilde{R}$. This trajectory only deviates from $\mathscr{I^+}$ and $  \mathscr{I^-}$ in the middle of the diagram. This makes the boundaries of $  \mathscr{I^+}$ and $  \mathscr{I^-}$ and idealization of a real observer's trajectory that is at a very far distance from the origin.
  
   \begin{figure}[!h]
               \centering
  \begin{tikzpicture}[x=0.75pt,y=0.75pt,yscale=-1,xscale=1]
  %uncomment if require: \path (0,163); %set diagram left start at 0, and has height of 163
 \tikzstyle{arrow} = [thick,->,>=stealth]
  %Draw the legend on the side
 \draw [arrow] (-120,81.5) -- (-120,40);
  \draw [arrow]  (-120,81.5) -- (-80,81.5);
%Draw triangles
 \draw[dashed] (0,0) -- (0,163);
 \draw (0,0) -- (130,81.5);
  \draw (0,163) -- (130,81.5);
 %Draw Vertical curves
 \draw (0,0) .. controls (20,74) and (20,89) .. (0,163);
  \draw (0,0) .. controls (60,74) and (60,89) .. (0,163);
   \draw (0,0) .. controls (100,74) and (100,89) .. (0,163);
   
     \draw [blue,thick](0,0) .. controls (134,79) and (134,84) .. (0,163);
   %Draw Horizontal curves
     \draw (0,81.5) -- (130,81.5);
   \draw (0,40) .. controls (60,40) and (100,60) .. (130,81.5);
 \draw (0,123) .. controls (60,123) and (100,103) .. (130,81.5);
  %Draw light rays
  \draw [arrow] (0,61.5)--(50,31.5)
   \draw [arrow] (0,101.5)--(50,131.5)
   
%Draw Light Cone

\draw (42,55) ellipse (10 and 2.5);
\draw (42,75) ellipse (10 and 2.5);
\draw (32,55)--(52,75);
\draw (52,55)--(32,75);
  % Label the points
  \draw  (-80,81.5)node [anchor=north west][inner sep=0.75pt]    {$\widetilde{R}$};
  \draw (-120,20) node [anchor=north west][inner sep=0.75pt]    {$ \widetilde{T}$};
  
  \draw (0,-20) node [anchor=north west][inner sep=0.75pt]    {$i^+\,:timelike \,future\, infinity$};
   \draw (0,173) node [anchor=north west][inner sep=0.75pt]    {$ i^-\,:timelike \,past\, infinity$};
    \draw (143,71.5) node [anchor=north west][inner sep=0.75pt]    {$i^0\,spacelike \,infinity$};
     \draw (60,131.5) node [anchor=north west][inner sep=0.75pt]  {$\mathscr{I^-}$}
      \draw (60,21.5) node [anchor=north west][inner sep=0.75pt]  {$\mathscr{I^+}$}
% Draw the points
\filldraw [black] (0,0) circle (2pt);
% Draw the points
\filldraw [black] (0,163) circle (2pt);
% Draw the points
\filldraw [black] (130,81.5) circle (2pt);
  \end{tikzpicture}
  \caption{Conformal Diagram of Minkowski Space: Restricted by $|\widetilde{T}|+\widetilde{R}<\pi$ }
  \label{fig:lineup}
              \end{figure}     
  
  
  Lets look at another conformal diagram, the one for the Schwarzschild Black Hole.
  In a previous class, the Kruskal the conveniently had light rays at $45^\circ$.
  \newline
  \begin{figure}[!h]
               \centering
  \begin{tikzpicture}[x=0.75pt,y=0.75pt,yscale=-1,xscale=1]
  %uncomment if require: \path (0,163); %set diagram left start at 0, and has height of 163
   \tikzstyle{arrow} = [thick,->,>=stealth]
    
 \draw [arrow](50,100) -- (50,0);
  \draw [arrow](0,50) -- (100,50);
  \draw [arrow](50,0)--(50,100);
  \draw [arrow] (100,50)--(0,50);
 
   \draw (0,100) -- (100,0);%III to I
    \draw (0,0) -- (100,100);%II to IV 
    
      \draw [arrow](80,20) .. controls (90,25) and (100,25) .. (110,15);%EH arrow
%Draw Light Cone

\draw [fill= gray!25](65,25) ellipse (10 and 2.5);
\draw [fill =gray!25](65,45) ellipse (10 and 2.5);
\draw (55,25)--(75,45);
\draw (75,25)--(55,45);
%Draw Curves
 \draw [dashed](0,0) .. controls (40,20) and (60,20) .. (100,0);%top
  \draw [dashed](0,100) .. controls (40,80) and (60,80) .. (100,100);%bottom
  % Label the points
  \draw  (105,40)node [anchor=north west][inner sep=0.75pt]    {$R$};
  \draw (50,-20) node [anchor=north west][inner sep=0.75pt]    {$ T$};
\draw (0,100) node [anchor=north west][inner sep=0.75pt]    {$r=0$};
\draw (0,-15) node [anchor=north west][inner sep=0.75pt]    {$r=0$};
\draw (110,0) node [anchor=north west][inner sep=0.75pt]    {$\small Event\,Horizon\,r=2M$};
  \end{tikzpicture}
  \caption{SBH digram in Kruskal coordinates}
  \label{fig:lineup}
              \end{figure}  
              
              \newpage
              However, these Kruskal coordinates still need some changes to create the conformal diagram. These coordinates need to be compactified. 
              U and V are rescalings of u and v coordinates
              \begin{equation}
                 \begin{align}
                 u=t-r_* && v=t+r_* \\
                 T=\frac{V+U}{2}&&  R=\frac{V-U}{2}\\
                 U=-e^{\frac{-u}{4m}}&&V=e^{\frac{v}{4m}}\\
             \end{align}
              \end{equation}
              Define compactified coordinates
              \begin{equation}
              \begin{align}
                 V=tan\widetilde{V} && U=tan\widetilde{U}
                 \end{align}
              \end{equation}
              The same steps can be taken to transform the metric, identify a conformal factor, multiplying $d\widetilde{V}d\widetilde{U}$, then finally define new coordinates, 
              \begin{equation}
                  \widetilde{T}=\widetilde{V}+\widetilde{U}\qquad \widetilde{R}=\widetilde{V}-\widetilde{U}
              \end{equation}
              
              then get rid of conformal factor, to look at the conformally related metric. 
              \begin{equation}
                  d\widetilde{s}^2=-d\widetilde{T}^2+d\widetilde{R}^2+(function\, of\, \widetilde{T}, \widetilde{R})d\Omega^2
              \end{equation}
              Now the conformal diagram for Schwarzschild can be draw. Here, $H^+$ and $H^-$ are the future and past horizons respectively, and all lines are still at $45^\circ$.
                \begin{figure}[!h]
               \centering
  \begin{tikzpicture}[x=0.75pt,y=0.75pt,yscale=-1,xscale=1]
  %uncomment if require: \path (0,163); %set diagram left start at 0, and has height of 163
 \tikzstyle{arrow} = [thick,->,>=stealth]
  %Draw the legend on the side
 \draw [arrow] (-120,81.5) -- (-120,40);
  \draw [arrow]  (-120,81.5) -- (-80,81.5);
  %r=0 arrow
  \draw [arrow] (98,-12) -- (88,15);
  
  \tikzset{snake it/.style={decorate, decoration=snake}};
 %Draw overly complicated geometric object 
  % Draw the points
\filldraw [black] (0,60) circle (1pt);
\filldraw [black] (171,60) circle (1pt);
\filldraw [black] (50,20) circle (1pt);
\filldraw [black] (121,20) circle (1pt);
\filldraw [black] (50,100) circle (1pt);
\filldraw [black] (121,100) circle (1pt);
  %Draw lines
  \draw (0,60) --(50,20);
  \draw (0,60)--(50,100);
  \draw [snake it](50,20)--(121,20);
  \draw (50,20)--(121,100);
  \draw (121,20)--(171,60);
   \draw (121,20)--(50,100);
   \draw [snake it] (50,100)--(121,100);
   \draw (121,100)--(171,60) ;
  %Labelling
   \draw  (-80,81.5)node [anchor=north west][inner sep=0.75pt]  {$\widetilde{R}$};
  \draw (-120,20) node [anchor=north west][inner sep=0.75pt]    {$ \widetilde{T}$};
  
   \draw  (-20,60) node [anchor=north west][inner sep=0.75pt] {$i^0$};
    \draw  (171,60) node [anchor=north west][inner sep=0.75pt] {$i^0$};
     \draw  (50,0) node [anchor=north west][inner sep=0.75pt] {$i^+$};
      \draw  (121,0)node [anchor=north west][inner sep=0.75pt] {$i^+$};
       \draw  (50,100) node [anchor=north west][inner sep=0.75pt] {$i^-$};
        \draw  (121,100) node [anchor=north west][inner sep=0.75pt] {$i^-$};
     
        \draw  (43,35) node [anchor=north west][inner sep=0.75pt] {$H^+$};
            \draw  (105,35) node [anchor=north west][inner sep=0.75pt] {$H^+$};
                \draw  (43,70) node [anchor=north west][inner sep=0.75pt] {$H^-$};
                    \draw  (105,70) node [anchor=north west][inner sep=0.75pt] {$H^-$};
                    
                     \draw  (100,-20) node [anchor=north west][inner sep=0.75pt] {$r=0$};
                     \draw  (160,20) node [anchor=north west][inner sep=0.75pt] {$ \mathscr{I^+}$};
                     \draw  (160,90) node [anchor=north west][inner sep=0.75pt] {$ \mathscr{I^-}$};
  \end{tikzpicture}
  \caption{Conformal Diagram for Schwarzschild }
  \label{fig:lineup}
              \end{figure}  
              
            
              Looking at Figure 7, The conformal diagram for a spherically symmetric star, the region representing the interior of the spherically symmetric star, the interior of the triangle, is the same as for the Schwarzschild Black hole. The shaded region represents the stellar interior. \newpage
               \begin{figure}[!h]
               \centering
  \begin{tikzpicture}[x=0.75pt,y=0.75pt,yscale=-1,xscale=1]
  %uncomment if require: \path (0,163); %set diagram left start at 0, and has height of 163
  \tikzstyle{arrow} = [thick,->,>=stealth]
  %Draw the legend on the side
 \draw [arrow] (-120,81.5) -- (-120,40);
  \draw [arrow]  (-120,81.5) -- (-80,81.5);
%Draw triangles
 \draw[dashed] (0,0) -- (0,163);
 \draw (0,0) -- (150,81.5);
  \draw (0,163) -- (150,81.5);
     %make shaded region
    \draw [gray] (2,8)--(2,158);
            \draw [gray] (4,14)--(4,152);
                \draw [gray] (6,22)--(6,142);
                    \draw [gray] (8,30)--(8,132);
                    \draw [gray] (10,40)--(10,120);
                    \draw [gray] (12,50)--(12,110);
                    \draw [gray] (14,60)--(14,99);
                   
  
 %Draw Vertical curves
 \draw (0,0) .. controls (20,74) and (20,89) .. (0,163);
 
  
  % Label the points
  \draw  (-80,81.5)node [anchor=north west][inner sep=0.75pt]    {$\widetilde{R}$};
  \draw (-120,20) node [anchor=north west][inner sep=0.75pt]    {$ \widetilde{T}$};
  
  \draw (0,-20) node [anchor=north west][inner sep=0.75pt]    {$i^+$};
   \draw (0,183) node [anchor=north west][inner sep=0.75pt]    {$ i^-$};
    \draw (163,81.5) node [anchor=north west][inner sep=0.75pt]    {$i^0$};
    
        \draw (100,30) node [anchor=north west][inner sep=0.75pt]    {$\mathscr{F^+}$};
            \draw (100,120) node [anchor=north west][inner sep=0.75pt]    {$\mathscr{F^-}$};
% Draw the points
\filldraw [black] (0,0) circle (2pt);
% Draw the points
\filldraw [black] (0,163) circle (2pt);
% Draw the points
\filldraw [black] (150,81.5) circle (2pt);
  \end{tikzpicture}
  \caption{What a conformal diagram will look like for our coordinates }
  \label{fig:lineup}
              \end{figure}     
               If a star were to collapse to form a black hole,the following conformal diagram could describe the space. Unlike Figure 7, this conformal diagram has a diagonal line reaching $i^+$, representing the event horizon acting as a null surface when the star collapses to a black hole. Then the surface of the star (shaded region), enters into the event horizon. Eventually the star hits a singularity, represented by the horizontal wavy line.
               \begin{figure}[!h]
               \centering
  \begin{tikzpicture}[x=0.75pt,y=0.75pt,yscale=-1,xscale=1]
  %uncomment if require: \path (0,163); %set diagram left start at 0, and has height of 163
  \tikzstyle{arrow} = [thick,->,>=stealth]
 
   \tikzset{snake it/.style={decorate, decoration=snake}};
  %Draw the legend on the side
 \draw [arrow] (-120,81.5) -- (-120,40);
  \draw [arrow]  (-120,81.5) -- (-80,81.5);
%Draw triangles
 \draw[dashed,name path=1] (0,40) -- (0,163);
 \draw (73,40) -- (140,81.5);
  \draw (0,163) -- (140,81.5);
    \draw (0,80) -- (73,40);
   \draw [snake it](0,40) -- (73,40);
  
   %make shaded region
    \draw [gray] (2,38)--(2,158);
            \draw [gray] (4,37)--(4,152);
                \draw [gray] (6,38)--(6,142);
                    \draw [gray] (8,40)--(8,132);
                    \draw [gray] (10,43)--(10,120);
                    \draw [gray] (12,43)--(12,110);
                    \draw [gray] (14,40)--(14,99);
                    \draw [gray] (16,44)--(16,90);
 %Draw Vertical curves
 \draw [name path=2] (15,40) .. controls (20,74) and (20,89) .. (0,163);
 
  
  % Label the points
  \draw  (-80,81.5)node [anchor=north west][inner sep=0.75pt]    {$\widetilde{R}$};
  \draw (-120,20) node [anchor=north west][inner sep=0.75pt]    {$ \widetilde{T}$};
  
  \draw (60,20) node [anchor=north west][inner sep=0.75pt]    {$i^+$};
   \draw (0,183) node [anchor=north west][inner sep=0.75pt]    {$ i^-$};
    \draw (153,81.5) node [anchor=north west][inner sep=0.75pt]    {$i^0$};
            \draw (100,30) node [anchor=north west][inner sep=0.75pt]    {$\mathscr{I^+}$};
            \draw (100,120) node [anchor=north west][inner sep=0.75pt]    {$\mathscr{I^-}$};
              \draw (40,61.5) node [anchor=north west][inner sep=0.75pt]    {$H^+$};
% Draw the points
\filldraw [black] (73,40) circle (2pt);
% Draw the points
\filldraw [black] (0,163) circle (2pt);
% Draw the points
\filldraw [black] (140,81.5) circle (2pt);


  \end{tikzpicture}
  \caption{Conformal Diagram of Star Collapsing to Form Black Hole }
  \label{fig:lineup}
              \end{figure}  
\end{document}